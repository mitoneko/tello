\documentclass[a4paper, 12pt]{ltjsarticle}
\usepackage[pdfencoding=auto,colorlinks,linkcolor=blue,urlcolor=blue]{hyperref}
\usepackage[top=1cm,bottom=1cm,left=2cm,right=1cm,includefoot]{geometry}
\usepackage{listings, listings-rust}
\lstset{
    language = Rust,
    breaklines = true,
    breakindent = 8pt,
    basicstyle = \ttfamily\small,
    frame = TBrl,
    framesep = 5pt,
    numbers = left,
    stepnumber = 1,
    numberstyle = \tiny,
    tabsize = 4,
    captionpos = t
}
\usepackage{graphicx}
\usepackage{multirow}

\title{Tello 飛行実験ソース}
\author{美都}

\begin{document}
\maketitle
\tableofcontents
\clearpage

\section {Tello本体 仕様}
\subsection {Telloの制御}
制御は、192.168.10.1:8889に対して、UDPでコントロールコマンドをテキストで送る。

コントロールコマンド列は、次のようになる。
\begin{description}
    \item[制御コマンド]\mbox{}\\
        コントローラーの制御コマンド群。レスポンスは、ok/error。
        \begin{table}[h]
            \hspace{3\zw}
            \begin{tabular}{ll}
                \hline
                コマンド & 動作 \\
                \hline \hline
                command & SDK制御ON \\ \hline
                streamon & ビデオストリーム オン \\ \hline
                streamoff & ビデオストリーム オフ \\ \hline
                emergency & 緊急停止 \\ \hline
                mon & ミッションパッド有効 \\ \hline
                moff & ミッションパッド無効 \\ \hline
                mdirection \emph{x} & 
                    \begin{minipage}{0.45\textwidth}
                        \vspace{0.5\zh}
                        ミッションパッドの検知モード設定
                        \vspace{-1\zh}
                        \begin{quote}
                            \begin{description}
                                \item[x=0]下方向のみ有効
                                \item[x=1]前方のみ有効
                                \item[x=2]下・前方の両方が有効
                            \end{description}
                        \end{quote}
                        \vspace{-1\zh}
                        x=0,1の時、ステータス取得が20Hz。\\
                        x=2の時、ステータス取得が10Hz。
                        \vspace{0.5\zh}
                    \end{minipage} \\ \hline
                ap \emph{ssid} \emph{pass} &
                \begin{tabular}{l}
                    TelloのWi-Fiを端末モードに切り替える \\
                    ssidとpassには、APのssidとパスワードを指定する。\\
                \end{tabular} \\ \hline
                wifi \emph{ssid} \emph{pass} &
                    Telloのssidとpassを変更する。\\ \hline
            \end{tabular}
        \end{table}
    \item[離着陸]\mbox{}\\
        離着陸を行う。レスポンスは、ok/error。
        \begin{table}[h]
            \hspace{3\zw}
            \begin{tabular}{ll}
                \hline
                コマンド & 動作 \\ \hline \hline
                takeoff & 離陸する。\\ \hline
                land & 着陸する。\\ \hline
            \end{tabular}
        \end{table}
        \clearpage
    \item[単純動作コマンド]\mbox{}\\
        移動のためのコマンド群。レスポンスは、ok/error。
        \begin{table}[h]
            \hspace{3\zw}
            \begin{tabular}{ll}
                \hline
                コマンド & 動作 \\ \hline \hline
                up \emph{x} & \emph{x}cm上昇する。\(20<=x<=500\)。\\ \hline
                down \emph{x} & \emph{x}cm下降する。 \(20<=x<=500\)。 \\ \hline
                forward \emph{x} & \emph{x}cm前進する。 \(20<=x<=500\)。 \\ \hline
                back \emph{x} & \emph{x}cm後退する。 \(20<=x<=500\)。 \\ \hline
                left \emph{x} & \emph{x}cm左に進む。 \(20<=x<=500\)。 \\ \hline
                right \emph{x} & \emph{x}cm右に進む。 \(20<=x<=500\)。 \\ \hline
                cw \emph{x} & \(x^\circ\)時計回りに旋回する。\(1<=x<=360\) \\ \hline
                ccw \emph{x} & \(x^\circ\)半時計回りに旋回する。\(1<=x<=360\) \\ \hline
                speed \emph{x} & 移動速度を\emph{x}(cm/s)に設定する。\(10<=x<=100\) \\ \hline
                stop & その場でホバリングする。\\ \hline
            \end{tabular}
        \end{table}
    \item[複合動作コマンド]\mbox{}\\
        移動のためのコマンド群。レスポンスは、ok/error。\\
        全てのコマンドで、\(|x_n|,|y_n|,|z_n|\)は、同時に20以下になってはいけない。さらに、各々の値は、
        \begin{eqnarray}
            0 < x_n,y_n,z_n < 500 (cm) \nonumber \\
            10 < speed < 100 (cm/s) \nonumber
        \end{eqnarray}
        を満たす。

        \begin{table}[h]
            \hspace{3\zw}
            \begin{tabular}{ll}
                \hline
                コマンド & 動作 \\ \hline \hline
                flip \emph{x} &
                \begin{minipage}{0.45\textwidth}
                    \vspace{0.5\zh}
                    \emph{x}で示す方向に宙返りする。
                    \vspace{-1\zh}
                    \begin{quote}
                        \begin{description}
                            \item["l"] 左
                            \item["r"] 右
                            \item["f"] 前方
                            \item["b"] 後方
                        \end{description}
                    \end{quote}
                    \vspace{-0.5\zh}
                \end{minipage} \\ \hline
                go \emph{x} \emph{y} \emph{z} \emph{speed} &
                \begin{minipage}{0.45\textwidth}
                    \vspace{0.5\zh}
                    現位置を基準とし、\((x,y,z)\)へ\(speed(cm/s)\)で移動する。
                    \vspace{0.5\zh}
                \end{minipage} \\ \hline
                courve \(x_1\) \(y_1\) \(z_1\) \(x_2\) \(y_2\) \(z_2\) \(speed\) &
                \begin{minipage}{0.45\textwidth}
                    \vspace{0.5\zh}
                    座標\((x_1,y_1,z_1)\)を経由して、\((x_2,y_2,z_2)\)へ\(speed\)(cm/s)で移動する。
                    移動経路の半径rは、\(0.5 < r < 10\)mとする。条件を満たさない場合、errorを返す。
                    \vspace{0.5\zh}
                \end{minipage} \\ \hline
            \end{tabular}
        \end{table}
        \clearpage
    \item[ミッションパッドコマンド]\mbox{}\\
        ミッションパッド関係のコマンド群。\\
        コマンド中\(mid_n\)は、ミッションパッドIDを意味する。書式は、"m1-m8"となる。\\
        レスポンスは、ok/error。\\
        全てのコマンドで、\(|x_n|,|y_n|,|z_n|\)は、同時に20以下になってはいけない。さらに、各々の値は、
        \begin{eqnarray}
            0 < x_n,y_n,z_n < 500 (cm) \nonumber \\
            10 < speed < 100 (cm/s) \nonumber
        \end{eqnarray}
        を満たす。

        \begin{table}[h]
            \hspace{3\zw}
            \begin{tabular}{ll}
                \hline
                コマンド & 動作 \\ \hline \hline
                 go \(x\) \(y\) \(z\) \(speed\) \(mid\) &
                \begin{minipage}{0.45\textwidth}
                    \vspace{0.5\zh}
                    midのパッドを基点として、\(speed\)(cm/s)で、\((x,y,z)\)の位置に移動する。
                    \vspace{0.5\zh}
                \end{minipage} \\ \hline
                \begin{minipage}{0.25\textwidth}
                    courve \(x_1\) \(y_1\) \(z_1\) \(x_2\) \(y_2\) \(z_2\) \(speed\) \(mid\) 
                \end{minipage} &
                \begin{minipage}{0.45\textwidth}
                    \vspace{0.5\zh}
                    ミッションパッド\(mid\)を基点として、座標\((x_1,y_1,z_1)\)を経由して、\((x_2,y_2,z_2)\)へ\(speed\)(cm/s)で移動する。
                    移動経路の半径rは、\(0.5 < r < 10\)mとする。条件を満たさない場合、errorを返す。
                    \vspace{0.5\zh}
                \end{minipage} \\ \hline
                \begin{minipage}{0.25\textwidth}
                    jump \(x\) \(y\) \(z\) \(speed\) \(yaw\) \(mid_1\) \(mid_2\) 
                \end{minipage} &
                \begin{minipage}{0.45\textwidth}
                    \vspace{0.5\zh}
                    ミッションパッド\(mid_1\)より、\(mid_2\)へ、\((x,y,z)\)を経由して移動し、\(yaw^\circ\)旋回する。
                    \vspace{0.5\zh}
                \end{minipage} \\ \hline
            \end{tabular}
        \end{table}
    \item[プロポコマンド]\mbox{}\\
        プロポ操作のコマンド。\\
        各動作方向のチャンネルの操作量を指定する。

        \begin{table}[h]
            \hspace{3\zw}
            \begin{tabular}{ll}
                \hline
                コマンド & 動作 \\ \hline \hline
                rc \(a\) \(b\) \(c\) \(d\) &
                \begin{minipage}{0.45\textwidth}
                    \vspace{0.5\zh}
                    \begin{description}
                        \item["a"] 左右 
                        \item["b"] 前後
                        \item["c"] 上下
                        \item["d"] 旋回
                    \end{description}
                    \( -100 <= a,b,c,d <= 100 \)
                    \vspace{0.5\zh}
                \end{minipage} \\ \hline
            \end{tabular}
        \end{table}
        \clearpage
    \item[問い合わせコマンド]\mbox{} \\
        各種問い合わせコマンド。\\
        レスポンスは、問い合わせの結果。
        \begin{table}[h]
            \hspace{3\zw}
            \begin{tabular}{lll}
                コマンド & 動作 & レスポンス \\ \hline \hline
                speed? & 現在の速度(cm/s) & 10-100 \\ \hline
                battery? & バッテリーの残量 & 0-100 \\ \hline
                time? & 今回のフライト時間 & 秒数 \\ \hline
                wifi? & Wi-Fi電波のSNR比 & SNR値 \\ \hline
                sdk? & SDKバージョン番号 & バージョン(整数2桁) \\ \hline
                sn? & TELLOのシリアル番号 & シリアル(文字列10文字) \\ \hline
            \end{tabular}
        \end{table}
\end{description}
\clearpage


\section {lib.rs}
\lstinputlisting[title={lib.rs}]{src/lib.rs}
\clearpage

\section {コントローラー}
Telloのコントローロールを司る。
\lstinputlisting[title={control.rs}]{src/control.rs}
\clearpage

\section {エラークラス}
エラー処理クラス。ライブラリーの全てのエラーを包含する。
\lstinputlisting[title={error.rs}]{src/error.rs}
\clearpage

\section {ステータスモジュール}
Telloのステータスを取得するための処理。
\subsection {モジュールトップ}
\lstinputlisting[title={mod.rs}]{src/status/mod.rs}
\clearpage

\subsection {データクラス}
ステータスデータを表すクラス。FromStrを実装し、UDPからの受信データに対して、parseが可能。
\lstinputlisting[title={data.rs}]{src/status/data.rs}
\clearpage

\subsection {マネージャクラス}
UDP通信を管理し、ステータスの取得を可能とする。
\lstinputlisting[title={manager.rs}]{src/status/manager.rs}
\clearpage

\end{document}
